\section{Language}

A \emph{language} $\Lang$ consists of a set \textdom{Expr} of \emph{expressions} (metavariable $\expr$), a set \textdom{Val} of \emph{values} (metavariable $\val$), and a set \textdom{State} of \emph{states} (metvariable $\state$) such that
\begin{itemize}
\item There exist functions $\ofval : \textdom{Val} \to \textdom{Expr}$ and $\toval : \textdom{Expr} \pfn \textdom{val}$ (notice the latter is partial), such that
\begin{mathpar} {\All \expr, \val. \toval(\expr) = \val \Ra \ofval(\val) = \expr} \and {\All\val. \toval(\ofval(\val)) = \val} 
\end{mathpar}
\item There exists a \emph{primitive reduction relation} \[(-,- \step -,-,-) \subseteq \textdom{Expr} \times \textdom{State} \times \textdom{Expr} \times \textdom{State} \times (\textdom{Expr} \uplus \set{\bot})\]
  We will write $\expr_1, \state_1 \step \expr_2, \state_2$ for $\expr_1, \state_1 \step \expr_2, \state_2, \bot$. \\
  A reduction $\expr_1, \state_1 \step \expr_2, \state_2, \expr_\f$ indicates that, when $\expr_1$ reduces to $\expr_2$, a \emph{new thread} $\expr_\f$ is forked off.
\item All values are stuck:
\[ \expr, \_ \step  \_, \_, \_ \Ra \toval(\expr) = \bot \]
\end{itemize}

\begin{defn}
  An expression $\expr$ and state $\state$ are \emph{reducible} (written $\red(\expr, \state)$) if
  \[ \Exists \expr_2, \state_2, \expr_\f. \expr,\state \step \expr_2,\state_2,\expr_\f \]
\end{defn}

\begin{defn}
  An expression $\expr$ is said to be \emph{atomic} if it reduces in one step to a value:
  \[ \All\state_1, \expr_2, \state_2, \expr_\f. \expr, \state_1 \step \expr_2, \state_2, \expr_\f \Ra \Exists \val_2. \toval(\expr_2) = \val_2 \]
\end{defn}

\begin{defn}[Context]
  A function $\lctx : \textdom{Expr} \to \textdom{Expr}$ is a \emph{context} if the following conditions are satisfied:
  \begin{enumerate}[itemsep=0pt]
  \item $\lctx$ does not turn non-values into values:\\
    $\All\expr. \toval(\expr) = \bot \Ra \toval(\lctx(\expr)) = \bot $
  \item One can perform reductions below $\lctx$:\\
    $\All \expr_1, \state_1, \expr_2, \state_2, \expr_\f. \expr_1, \state_1 \step \expr_2,\state_2,\expr_\f \Ra \lctx(\expr_1), \state_1 \step \lctx(\expr_2),\state_2,\expr_\f $
  \item Reductions stay below $\lctx$ until there is a value in the hole:\\
    $\All \expr_1', \state_1, \expr_2, \state_2, \expr_\f. \toval(\expr_1') = \bot \land \lctx(\expr_1'), \state_1 \step \expr_2,\state_2,\expr_\f \Ra \Exists\expr_2'. \expr_2 = \lctx(\expr_2') \land \expr_1', \state_1 \step \expr_2',\state_2,\expr_\f $
  \end{enumerate}
\end{defn}

\subsection{Concurrent language}

For any language $\Lang$, we define the corresponding thread-pool semantics.

\paragraph{Machine syntax}
\[
	\tpool \in \textdom{ThreadPool} \eqdef \bigcup_n \textdom{Expr}^n
\]

\judgment[Machine reduction]{\cfg{\tpool}{\state} \step
  \cfg{\tpool'}{\state'}}
\begin{mathpar}
\infer
  {\expr_1, \state_1 \step \expr_2, \state_2, \expr_\f \and \expr_\f \neq \bot}
  {\cfg{\tpool \dplus [\expr_1] \dplus \tpool'}{\state_1} \step
     \cfg{\tpool \dplus [\expr_2] \dplus \tpool' \dplus [\expr_\f]}{\state_2}}
\and\infer
  {\expr_1, \state_1 \step \expr_2, \state_2}
  {\cfg{\tpool \dplus [\expr_1] \dplus \tpool'}{\state_1} \step
     \cfg{\tpool \dplus [\expr_2] \dplus \tpool'}{\state_2}}
\end{mathpar}

\clearpage
\section{Program Logic}

\ralf{TODO: Right now, this is a dump of all the things that moved out of the base...}

To instantiate Iris, you need to define the following parameters:
\begin{itemize}
\item A language $\Lang$, and
\item a locally contractive bifunctor $\iFunc : \COFEs \to \CMRAs$ defining the ghost state, such that for all COFEs $\cofe$, the CMRA $\iFunc(A)$ has a unit. (By \lemref{lem:cmra-unit-total-core}, this means that the core of $\iFunc(\cofe)$ is a total function.)
\end{itemize}

We will write $\pvs[\term] \prop$ for $\pvs[\term][\term] \prop$.
If we omit the mask, then it is $\top$ for weakest precondition $\wpre\expr{\Ret\var.\prop}$ and $\emptyset$ for primitive view shifts $\pvs \prop$.
%FIXME $\top$ is not a term in the logic. Neither is any of the operations on masks that we use in the rules for weakestpre.

Some propositions are \emph{timeless}, which intuitively means that step-indexing does not affect them.
This is a \emph{meta-level} assertion about propositions, defined as follows:

\[ \vctx \proves \timeless{\prop} \eqdef \vctx\mid\later\prop \proves \prop \lor \later\FALSE \]

\paragraph{Metavariable conventions.}
We introduce additional metavariables ranging over terms and generally let the choice of metavariable indicate the term's type:
\[
\begin{array}{r|l}
 \text{metavariable} & \text{type} \\\hline
  \term, \termB & \text{arbitrary} \\
  \val, \valB & \textlog{Val} \\
  \expr & \textlog{Expr} \\
  \state & \textlog{State} \\
\end{array}
\qquad\qquad
\begin{array}{r|l}
 \text{metavariable} & \text{type} \\\hline
  \iname & \textlog{InvName} \\
  \mask & \textlog{InvMask} \\
  \melt, \meltB & \textlog{M} \\
  \prop, \propB, \propC & \Prop \\
  \pred, \predB, \predC & \type\to\Prop \text{ (when $\type$ is clear from context)} \\
\end{array}
\]

\begin{mathpar}
\infer
{\text{$\term$ or $\term'$ is a discrete COFE element}}
{\timeless{\term =_\type \term'}}

\infer
{\text{$\melt$ is a discrete COFE element}}
{\timeless{\ownGGhost\melt}}

\infer
{\text{$\melt$ is an element of a discrete CMRA}}
{\timeless{\mval(\melt)}}

\infer{}
{\timeless{\ownPhys\state}}

\infer
{\vctx \proves \timeless{\propB}}
{\vctx \proves \timeless{\prop \Ra \propB}}

\infer
{\vctx \proves \timeless{\propB}}
{\vctx \proves \timeless{\prop \wand \propB}}

\infer
{\vctx,\var:\type \proves \timeless{\prop}}
{\vctx \proves \timeless{\All\var:\type.\prop}}

\infer
{\vctx,\var:\type \proves \timeless{\prop}}
{\vctx \proves \timeless{\Exists\var:\type.\prop}}
\end{mathpar}

\begin{mathpar}
\infer[pvs-intro]
{}{\prop \proves \pvs[\mask] \prop}

\infer[pvs-mono]
{\prop \proves \propB}
{\pvs[\mask_1][\mask_2] \prop \proves \pvs[\mask_1][\mask_2] \propB}

\infer[pvs-timeless]
{\timeless\prop}
{\later\prop \proves \pvs[\mask] \prop}

\infer[pvs-trans]
{\mask_2 \subseteq \mask_1 \cup \mask_3}
{\pvs[\mask_1][\mask_2] \pvs[\mask_2][\mask_3] \prop \proves \pvs[\mask_1][\mask_3] \prop}

\infer[pvs-mask-frame]
{}{\pvs[\mask_1][\mask_2] \prop \proves \pvs[\mask_1 \uplus \mask_\f][\mask_2 \uplus \mask_\f] \prop}

\infer[pvs-frame]
{}{\propB * \pvs[\mask_1][\mask_2]\prop \proves \pvs[\mask_1][\mask_2] \propB * \prop}

\inferH{pvs-allocI}
{\text{$\mask$ is infinite}}
{\later\prop \proves \pvs[\mask] \Exists \iname \in \mask. \knowInv\iname\prop}

\inferH{pvs-openI}
{}{\knowInv\iname\prop \proves \pvs[\set\iname][\emptyset] \later\prop}

\inferH{pvs-closeI}
{}{\knowInv\iname\prop \land \later\prop \proves \pvs[\emptyset][\set\iname] \TRUE}

\inferH{pvs-update}
{\melt \mupd \meltsB}
{\ownGGhost\melt \proves \pvs[\mask] \Exists\meltB\in\meltsB. \ownGGhost\meltB}
\end{mathpar}

\paragraph{Laws of weakest preconditions.}
\begin{mathpar}
\infer[wp-value]
{}{\prop[\val/\var] \proves \wpre{\val}[\mask]{\Ret\var.\prop}}

\infer[wp-mono]
{\mask_1 \subseteq \mask_2 \and \var:\textlog{val}\mid\prop \proves \propB}
{\wpre\expr[\mask_1]{\Ret\var.\prop} \proves \wpre\expr[\mask_2]{\Ret\var.\propB}}

\infer[pvs-wp]
{}{\pvs[\mask] \wpre\expr[\mask]{\Ret\var.\prop} \proves \wpre\expr[\mask]{\Ret\var.\prop}}

\infer[wp-pvs]
{}{\wpre\expr[\mask]{\Ret\var.\pvs[\mask] \prop} \proves \wpre\expr[\mask]{\Ret\var.\prop}}

\infer[wp-atomic]
{\mask_2 \subseteq \mask_1 \and \physatomic{\expr}}
{\pvs[\mask_1][\mask_2] \wpre\expr[\mask_2]{\Ret\var. \pvs[\mask_2][\mask_1]\prop}
 \proves \wpre\expr[\mask_1]{\Ret\var.\prop}}

\infer[wp-frame]
{}{\propB * \wpre\expr[\mask]{\Ret\var.\prop} \proves \wpre\expr[\mask]{\Ret\var.\propB*\prop}}

\infer[wp-frame-step]
{\toval(\expr) = \bot \and \mask_2 \subseteq \mask_1}
{\wpre\expr[\mask]{\Ret\var.\prop} * \pvs[\mask_1][\mask_2]\later\pvs[\mask_2][\mask_1]\propB \proves \wpre\expr[\mask \uplus \mask_1]{\Ret\var.\propB*\prop}}

\infer[wp-bind]
{\text{$\lctx$ is a context}}
{\wpre\expr[\mask]{\Ret\var. \wpre{\lctx(\ofval(\var))}[\mask]{\Ret\varB.\prop}} \proves \wpre{\lctx(\expr)}[\mask]{\Ret\varB.\prop}}
\end{mathpar}

\paragraph{Lifting of operational semantics.}~
\begin{mathpar}
  \infer[wp-lift-step]
  {\mask_2 \subseteq \mask_1 \and
   \toval(\expr_1) = \bot}
  { {\begin{inbox} % for some crazy reason, LaTeX is actually sensitive to the space between the "{ {" here and the "} }" below...
        ~~\pvs[\mask_1][\mask_2] \Exists \state_1. \red(\expr_1,\state_1) \land \later\ownPhys{\state_1} * {}\\\qquad\qquad\qquad \later\All \expr_2, \state_2, \expr_\f. \left( (\expr_1, \state_1 \step \expr_2, \state_2, \expr_\f) \land \ownPhys{\state_2} \right) \wand \pvs[\mask_2][\mask_1] \wpre{\expr_2}[\mask_1]{\Ret\var.\prop} * \wpre{\expr_\f}[\top]{\Ret\any.\TRUE}  {}\\\proves \wpre{\expr_1}[\mask_1]{\Ret\var.\prop}
      \end{inbox}} }
\\\\
  \infer[wp-lift-pure-step]
  {\toval(\expr_1) = \bot \and
   \All \state_1. \red(\expr_1, \state_1) \and
   \All \state_1, \expr_2, \state_2, \expr_\f. \expr_1,\state_1 \step \expr_2,\state_2,\expr_\f \Ra \state_1 = \state_2 }
  {\later\All \state, \expr_2, \expr_\f. (\expr_1,\state \step \expr_2, \state,\expr_\f)  \Ra \wpre{\expr_2}[\mask_1]{\Ret\var.\prop} * \wpre{\expr_\f}[\top]{\Ret\any.\TRUE} \proves \wpre{\expr_1}[\mask_1]{\Ret\var.\prop}}
\end{mathpar}
Notice that primitive view shifts cover everything to their right, \ie $\pvs \prop * \propB \eqdef \pvs (\prop * \propB)$.

Here we define $\wpre{\expr_\f}[\mask]{\Ret\var.\prop} \eqdef \TRUE$ if $\expr_\f = \bot$ (remember that our stepping relation can, but does not have to, define a forked-off expression).

The adequacy statement concerning functional correctness reads as follows:
\begin{align*}
 &\All \mask, \expr, \val, \pred, \state, \melt, \state', \tpool'.
 \\&(\All n. \melt \in \mval_n) \Ra
 \\&( \ownPhys\state * \ownGGhost\melt \proves \wpre{\expr}[\mask]{x.\; \pred(x)}) \Ra
 \\&\cfg{\state}{[\expr]} \step^\ast
     \cfg{\state'}{[\val] \dplus \tpool'} \Ra
     \\&\pred(\val)
\end{align*}
where $\pred$ is a \emph{meta-level} predicate over values, \ie it can mention neither resources nor invariants.

Furthermore, the following adequacy statement shows that our weakest preconditions imply that the execution never gets \emph{stuck}: Every expression in the thread pool either is a value, or can reduce further.
\begin{align*}
 &\All \mask, \expr, \state, \melt, \state', \tpool'.
 \\&(\All n. \melt \in \mval_n) \Ra
 \\&( \ownPhys\state * \ownGGhost\melt \proves \wpre{\expr}[\mask]{x.\; \pred(x)}) \Ra
 \\&\cfg{\state}{[\expr]} \step^\ast
     \cfg{\state'}{\tpool'} \Ra
     \\&\All\expr'\in\tpool'. \toval(\expr') \neq \bot \lor \red(\expr', \state')
\end{align*}
Notice that this is stronger than saying that the thread pool can reduce; we actually assert that \emph{every} non-finished thread can take a step.

\subsection{Iris model}

\paragraph{Semantic domain of assertions.}
The first complicated task in building a model of full Iris is defining the semantic model of $\Prop$.
We start by defining the functor that assembles the CMRAs we need to the global resource CMRA:
\begin{align*}
  \textdom{ResF}(\cofe^\op, \cofe) \eqdef{}& \record{\wld: \mathbb{N} \fpfn \agm(\latert \cofe), \pres: \maybe{\exm(\textdom{State})}, \ghostRes: \iFunc(\cofe^\op, \cofe)}
\end{align*}
Above, $\maybe\monoid$ is the monoid obtained by adding a unit to $\monoid$.
(It's not a coincidence that we used the same notation for the range of the core; it's the same type either way: $\monoid + 1$.)
Remember that $\iFunc$ is the user-chosen bifunctor from $\COFEs$ to $\CMRAs$ (see~\Sref{sec:logic}).
$\textdom{ResF}(\cofe^\op, \cofe)$ is a CMRA by lifting the individual CMRAs pointwise.
Furthermore, since $\Sigma$ is locally contractive, so is $\textdom{ResF}$.

Now we can write down the recursive domain equation:
\[ \iPreProp \cong \UPred(\textdom{ResF}(\iPreProp, \iPreProp)) \]
$\iPreProp$ is a COFE defined as the fixed-point of a locally contractive bifunctor.
This fixed-point exists and is unique by America and Rutten's theorem~\cite{America-Rutten:JCSS89,birkedal:metric-space}.
We do not need to consider how the object is constructed. 
We only need the isomorphism, given by
\begin{align*}
  \Res &\eqdef \textdom{ResF}(\iPreProp, \iPreProp) \\
  \iProp &\eqdef \UPred(\Res) \\
	\wIso &: \iProp \nfn \iPreProp \\
	\wIso^{-1} &: \iPreProp \nfn \iProp
\end{align*}

We then pick $\iProp$ as the interpretation of $\Prop$:
\[ \Sem{\Prop} \eqdef \iProp \]


\paragraph{Interpretation of assertions.}
$\iProp$ is a $\UPred$, and hence the definitions from \Sref{sec:upred-logic} apply.
We only have to define the interpretation of the missing connectives, the most interesting bits being primitive view shifts and weakest preconditions.

\typedsection{World satisfaction}{\wsat{-}{-}{-} : 
	\Delta\textdom{State} \times
	\Delta\pset{\mathbb{N}} \times
	\textdom{Res} \nfn \SProp }
\begin{align*}
  \wsatpre(n, \mask, \state, \rss, \rs) & \eqdef \begin{inbox}[t]
    \rs \in \mval_{n+1} \land \rs.\pres = \exinj(\sigma) \land 
    \dom(\rss) \subseteq \mask \cap \dom( \rs.\wld) \land {}\\
    \All\iname \in \mask, \prop \in \iProp. (\rs.\wld)(\iname) \nequiv{n+1} \aginj(\latertinj(\wIso(\prop))) \Ra n \in \prop(\rss(\iname))
  \end{inbox}\\
	\wsat{\state}{\mask}{\rs} &\eqdef \set{0}\cup\setComp{n+1}{\Exists \rss : \mathbb{N} \fpfn \textdom{Res}. \wsatpre(n, \mask, \state, \rss, \rs \mtimes \prod_\iname \rss(\iname))}
\end{align*}

\typedsection{Primitive view-shift}{\mathit{pvs}_{-}^{-}(-) : \Delta(\pset{\mathbb{N}}) \times \Delta(\pset{\mathbb{N}}) \times \iProp \nfn \iProp}
\begin{align*}
	\mathit{pvs}_{\mask_1}^{\mask_2}(\prop) &= \Lam \rs. \setComp{n}{\begin{aligned}
            \All \rs_\f, k, \mask_\f, \state.& 0 < k \leq n \land (\mask_1 \cup \mask_2) \disj \mask_\f \land k \in \wsat\state{\mask_1 \cup \mask_\f}{\rs \mtimes \rs_\f} \Ra {}\\&
            \Exists \rsB. k \in \prop(\rsB) \land k \in \wsat\state{\mask_2 \cup \mask_\f}{\rsB \mtimes \rs_\f}
          \end{aligned}}
\end{align*}

\typedsection{Weakest precondition}{\mathit{wp}_{-}(-, -) : \Delta(\pset{\mathbb{N}}) \times \Delta(\textdom{Exp}) \times (\Delta(\textdom{Val}) \nfn \iProp) \nfn \iProp}

$\textdom{wp}$ is defined as the fixed-point of a contractive function.
\begin{align*}
  \textdom{pre-wp}(\textdom{wp})(\mask, \expr, \pred) &\eqdef \Lam\rs. \setComp{n}{\begin{aligned}
        \All &\rs_\f, m, \mask_\f, \state. 0 \leq m < n \land \mask \disj \mask_\f \land m+1 \in \wsat\state{\mask \cup \mask_\f}{\rs \mtimes \rs_\f} \Ra {}\\
        &(\All\val. \toval(\expr) = \val \Ra \Exists \rsB. m+1 \in \pred(\val)(\rsB) \land m+1 \in \wsat\state{\mask \cup \mask_\f}{\rsB \mtimes \rs_\f}) \land {}\\
        &(\toval(\expr) = \bot \land 0 < m \Ra \red(\expr, \state) \land \All \expr_2, \state_2, \expr_\f. \expr,\state \step \expr_2,\state_2,\expr_\f \Ra {}\\
        &\qquad \Exists \rsB_1, \rsB_2. m \in \wsat\state{\mask \cup \mask_\f}{\rsB_1 \mtimes \rsB_2 \mtimes \rs_\f} \land  m \in \textdom{wp}(\mask, \expr_2, \pred)(\rsB_1) \land {}&\\
        &\qquad\qquad (\expr_\f = \bot \lor m \in \textdom{wp}(\top, \expr_\f, \Lam\any.\Lam\any.\mathbb{N})(\rsB_2))
    \end{aligned}} \\
  \textdom{wp}_\mask(\expr, \pred) &\eqdef \mathit{fix}(\textdom{pre-wp})(\mask, \expr, \pred)
\end{align*}


\typedsection{Interpretation of program logic assertions}{\Sem{\vctx \proves \term : \Prop} : \Sem{\vctx} \nfn \iProp}

$\knowInv\iname\prop$, $\ownGGhost\melt$ and $\ownPhys\state$ are just syntactic sugar for forms of $\ownM{-}$.
\begin{align*}
	\knowInv{\iname}{\prop} &\eqdef \ownM{[\iname \mapsto \aginj(\latertinj(\wIso(\prop)))], \munit, \munit} \\
	\ownGGhost{\melt} &\eqdef \ownM{\munit, \munit, \melt} \\
	\ownPhys{\state} &\eqdef \ownM{\munit, \exinj(\state), \munit} \\
~\\
	\Sem{\vctx \proves \pvs[\mask_1][\mask_2] \prop : \Prop}_\gamma &\eqdef
	\textdom{pvs}^{\Sem{\vctx \proves \mask_2 : \textlog{InvMask}}_\gamma}_{\Sem{\vctx \proves \mask_1 : \textlog{InvMask}}_\gamma}(\Sem{\vctx \proves \prop : \Prop}_\gamma) \\
	\Sem{\vctx \proves \wpre{\expr}[\mask]{\Ret\var.\prop} : \Prop}_\gamma &\eqdef
	\textdom{wp}_{\Sem{\vctx \proves \mask : \textlog{InvMask}}_\gamma}(\Sem{\vctx \proves \expr : \textlog{Expr}}_\gamma, \Lam\val. \Sem{\vctx \proves \prop : \Prop}_{\gamma[\var\mapsto\val]})
\end{align*}


%%% Local Variables:
%%% mode: latex
%%% TeX-master: "iris"
%%% End:
