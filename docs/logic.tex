
\section{Parameters to the logic}

\begin{itemize}
% \item A set \textdom{Exp} of \emph{expressions} (metavariable $\expr$) with a
%   subset \textdom{Val} of values ($\val$).  We assume that if $\expr$ is an
%   expression then so is $\fork{\expr}$.  We moreover assume a value
%   \textsf{fRet} (giving the intended return value of a fork), and we assume that
%   \begin{align*}
%     \fork{\expr} &\notin \textdom{Val} \\
%     \fork{\expr_1} = \fork{\expr_2} &\implies \expr_1 = \expr_2
%   \end{align*}
\item A set $\textdom{Ectx}$ of \emph{evaluation contexts} ($\ectx$) that includes the empty context $[\; ]$,
  a plugging operation $\ectx[\expr]$ that produces an expression, and context composition $\circ$
  satisfying the following axioms:
  \begin{align*}
    [\; ][ \expr ] &= \expr \\
    \ectx_1[\ectx_2[\expr]] &= (\ectx_1 \circ \ectx_2) [\expr] \\
    \ectx_1[\expr] = \ectx_2[\expr] &\implies \ectx_1 = \ectx_2 \\
    \ectx[\expr_1] = \ectx[\expr_2] &\implies \expr_1 = \expr_2 \\
    \ectx_1 \circ \ectx_2 = [\; ] &\implies \ectx_1 = \ectx_2 = [\; ] \\
    \ectx[\expr] \in \textdom{Val} &\implies \ectx = [\;] \\
%    \ectx[\expr] = \fork{\expr'} &\implies \ectx = [\;]
  \end{align*}

\item A set \textdom{State} of shared machine states (\eg heaps), metavariable $\state$.
\item An \emph{atomic stepping relation} \[
  (- \step -) \subseteq (\textdom{State} \times \textdom{Exp}) \times (\textdom{State} \times \textdom{Exp})
\]
and notions of an expression to be \emph{reducible} or \emph{stuck}, such that
\begin{align*}
  \textlog{reducible}(\expr) &\iff \Exists \state, \expr_2, \state_2. \cfg{\state}{\expr} \step \cfg{\state_2}{\expr_2} \\
%  \textlog{stuck}(\expr) &\iff \All \ectx, \expr'. \expr = \ectx[\expr'] \implies
   \lnot \textlog{reducible}(\expr')
\end{align*}
and the following hold
% \begin{align*}
% &\textlog{stuck}(\fork{\expr})& \\
%  &\textlog{stuck}(\val)&\\
%  &\ectx[\expr] = \ectx'[\expr'] \implies \textlog{reducible}(\expr') \implies
%   \expr \notin \textdom{Val} \implies \Exists \ectx''. \ectx' = \ectx \circ \ectx'' &\mbox{(step-by-value)} \\
%  &\ectx[\expr] = \ectx'[\fork{\expr'}] \implies
%   \expr \notin \textdom{Val} \implies \Exists \ectx''. \ectx' = \ectx \circ \ectx'' &\mbox{(fork-by-value)} \\
% \end{align*}

\item A predicate \textlog{atomic} on expressions satisfying
  \begin{align*}
   &\textlog{atomic}(\expr) \implies \textlog{reducible}(\expr) &\\
   &\textlog{atomic}(\expr) \implies \cfg{\state}{\expr} \step \cfg{\state_2}{\expr_2} \implies \expr_2 \in \textdom{Val} &\mbox{(atomic-step)}
  \end{align*}


\item A commutative monoid with zero, $M$.
That is, a set $\mcar{M}$ with two distinguished elements $\mzero$ (zero, undefined) and $\munit$ (one, unit) and an operation $\mtimes$ (times, combine) such that
\begin{align*}
 \melt \mtimes \meltB &= \meltB \mtimes \melt \\
 \munit \mtimes \melt &= \melt \\
 (\melt \mtimes \meltB) \mtimes \meltC &= \melt \mtimes (\meltB \mtimes \meltC) \\
 \mzero \mtimes \melt &= \mzero \\
 \mzero &\neq \munit
\end{align*}
Let $\mcarp{M} \eqdef |\monoid| \setminus \{\mzero\}$.

\item Arbitrary additional types and terms.
\end{itemize}

\section{The concurrent language}

\paragraph{Machine syntax}
\[
	\tpool \in \textdom{ThreadPool} \eqdef \mathbb{N} \fpfn \textdom{Exp}
\]

\judgment{Machine reduction} {\cfg{\state}{\tpool} \step
  \cfg{\state'}{\tpool'}}
\begin{mathpar}
\infer
  {\cfg{\state}{\expr} \step \cfg{\state'}{\expr'}}
  {\cfg{\state}{\tpool [i \mapsto \ectx[\expr]]} \step
     \cfg{\state'}{\tpool [i \mapsto \ectx[\expr']]}}
% \and
% \infer
%   {}
%   {\cfg{\state}{\tpool [i \mapsto \ectx[\fork{\expr}]]} \step
%     \cfg{\state}{\tpool [i \mapsto \ectx[\textsf{fRet}]] [j \mapsto \expr]}}
\end{mathpar}

\section{Syntax}

\subsection{Grammar}\label{sec:grammar}

\paragraph{Signatures.}
We use a signature to account syntactically for the logic's parameters.
A \emph{signature} $\Sig = (\SigType, \SigFn)$ comprises a set
\[
	\SigType \supseteq \{ \textsort{Val}, \textsort{Exp}, \textsort{Ectx}, \textsort{State}, \textsort{Monoid}, \textsort{InvName}, \textsort{InvMask}, \Prop \}
\]
of base types (or base \emph{sorts}) and a set $\SigFn$ of typed function symbols.
This means that each function symbol has an associated \emph{arity} comprising a natural number $n$ and an ordered list of $n+1$ base types.
We write
\[
	\sigfn : \type_1, \dots, \type_n \to \type_{n+1} \in \SigFn
\]
to express that $\sigfn$ is a function symbol with the indicated arity.
\dave{Say something not-too-shabby about adequacy: We don't spell out what it means.}

\paragraph{Syntax.}
Iris syntax is built up from a signature $\Sig$ and a countably infinite set $\textdom{Var}$ of variables (ranged over by metavariables $x$, $y$, $z$):
\newcommand{\unitterm}{()}%
\newcommand{\unitsort}{1}%	\unit is bold.
\begin{align*}
  \term, \prop, \pred ::={}&
      x \mid
      \sigfn(\term_1, \dots, \term_n) \mid
      \unitterm \mid
      (\term, \term) \mid
      \pi_i\; \term \mid
      \Lam x.\term \mid
      \term\;\term  \mid
      \mzero \mid
      \munit \mid
      \term \mtimes \term \mid
\\&
    \FALSE \mid
    \TRUE \mid
    \term =_\sort \term \mid
    \prop \Ra \prop \mid
    \prop \land \prop \mid
    \prop \lor \prop \mid
    \prop * \prop \mid
    \prop \wand \prop \mid
\\&
    \MU \var. \pred  \mid
    \Exists \var:\sort. \prop \mid
    \All \var:\sort. \prop \mid
\\&
    \knowInv{\term}{\prop} \mid
    \ownGGhost{\term} \mid
    \ownPhys{\term} \mid
    \always\prop \mid
    {\later\prop} \mid
    \pvsA{\prop}{\term}{\term} \mid
    \dynA{\term}{\pred}{\term} \mid
    \timeless{\prop}
\\[0.4em]
  \sort ::={}&
      \type \mid
      \unitsort \mid
      \sort \times \sort \mid
      \sort \to \sort
\end{align*}
Recursive predicates must be \emph{guarded}: in $\MU \var. \pred$, the variable $\var$ can only appear under the later $\later$ modality.

\paragraph{Metavariable conventions.}
We introduce additional metavariables ranging over terms and generally let the choice of metavariable indicate the term's sort:
\[
\begin{array}{r|l}
 \text{metavariable} & \text{sort} \\\hline
  \term, \termB & \text{arbitrary} \\
  \val, \valB & \textsort{Val} \\
  \expr & \textsort{Exp} \\
  \ectx & \textsort{Ectx} \\
  \state & \textsort{State} \\
\end{array}
\qquad\qquad
\begin{array}{r|l}
 \text{metavariable} & \text{sort} \\\hline
  \iname & \textsort{InvName} \\
  \mask & \textsort{InvMask} \\
  \melt, \meltB & \textsort{Monoid} \\
  \prop, \propB, \propC & \Prop \\
  \pred, \predB, \predC & \sort\to\Prop \text{ (when $\sort$ is clear from context)} \\
\end{array}
\]

\paragraph{Variable conventions.}
We often abuse notation, using the preceding \emph{term} metavariables to range over (bound) \emph{variables}.
We omit type annotations in binders, when the type is clear from context.


\subsection{Types}\label{sec:types}

Iris terms are simply-typed.
The judgment $\vctx \proves_\Sig \wtt{\term}{\sort}$ expresses that, in signature $\Sig$ and variable context $\vctx$, the term $\term$ has sort $\sort$.
In giving the rules for this judgment, we omit the signature (which does not change).

A variable context, $\vctx = x_1:\sort_1, \dots, x_n:\sort_n$, declares a list of variables and their sorts.
In writing $\vctx, x:\sort$, we presuppose that $x$ is not already declared in $\vctx$.

\judgment{Well-typed terms}{\vctx \proves_\Sig \wtt{\term}{\sort}}
\begin{mathparpagebreakable}
%%% variables and function symbols
	\axiom{x : \sort \proves \wtt{x}{\sort}}
\and
	\infer{\vctx \proves \wtt{\term}{\sort}}
		{\vctx, x:\sort' \proves \wtt{\term}{\sort}}
\and
	\infer{\vctx, x:\sort', y:\sort' \proves \wtt{\term}{\sort}}
		{\vctx, x:\sort' \proves \wtt{\term[x/y]}{\sort}}
\and
	\infer{\vctx_1, x:\sort', y:\sort'', \vctx_2 \proves \wtt{\term}{\sort}}
		{\vctx_1, x:\sort'', y:\sort', \vctx_2 \proves \wtt{\term[y/x,x/y]}{\sort}}
\and
	\infer{
		\vctx \proves \wtt{\term_1}{\type_1} \and
		\cdots \and
		\vctx \proves \wtt{\term_n}{\type_n} \and
		\sigfn : \type_1, \dots, \type_n \to \type_{n+1} \in \SigFn
	}{
		\vctx \proves \wtt {\sigfn(\term_1, \dots, \term_n)} {\type_{n+1}}
	}
%%% products
\and
	\axiom{\vctx \proves \wtt{\unitterm}{\unitsort}}
\and
	\infer{\vctx \proves \wtt{\term}{\sort_1} \and \vctx \proves \wtt{\termB}{\sort_2}}
		{\vctx \proves \wtt{(\term,\termB)}{\sort_1 \times \sort_2}}
\and
	\infer{\vctx \proves \wtt{\term}{\sort_1 \times \sort_2} \and i \in \{1, 2\}}
		{\vctx \proves \wtt{\pi_i\,\term}{\sort_i}}
%%% functions
\and
	\infer{\vctx, x:\sort \proves \wtt{\term}{\sort'}}
		{\vctx \proves \wtt{\Lam x. \term}{\sort \to \sort'}}
\and
	\infer
	{\vctx \proves \wtt{\term}{\sort \to \sort'} \and \wtt{\termB}{\sort}}
	{\vctx \proves \wtt{\term\;\termB}{\sort'}}
%%% monoids
\and
	\axiom{\vctx \proves \wtt{\mzero}{\textsort{Monoid}}}
\and
	\axiom{\vctx \proves \wtt{\munit}{\textsort{Monoid}}}
\and
	\infer{\vctx \proves \wtt{\melt}{\textsort{Monoid}} \and \vctx \proves \wtt{\meltB}{\textsort{Monoid}}}
		{\vctx \proves \wtt{\melt \mtimes \meltB}{\textsort{Monoid}}}
%%% props and predicates
\\
	\axiom{\vctx \proves \wtt{\FALSE}{\Prop}}
\and
	\axiom{\vctx \proves \wtt{\TRUE}{\Prop}}
\and
	\infer{\vctx \proves \wtt{\term}{\sort} \and \vctx \proves \wtt{\termB}{\sort}}
		{\vctx \proves \wtt{\term =_\sort \termB}{\Prop}}
\and
	\infer{\vctx \proves \wtt{\prop}{\Prop} \and \vctx \proves \wtt{\propB}{\Prop}}
		{\vctx \proves \wtt{\prop \Ra \propB}{\Prop}}
\and
	\infer{\vctx \proves \wtt{\prop}{\Prop} \and \vctx \proves \wtt{\propB}{\Prop}}
		{\vctx \proves \wtt{\prop \land \propB}{\Prop}}
\and
	\infer{\vctx \proves \wtt{\prop}{\Prop} \and \vctx \proves \wtt{\propB}{\Prop}}
		{\vctx \proves \wtt{\prop \lor \propB}{\Prop}}
\and
	\infer{\vctx \proves \wtt{\prop}{\Prop} \and \vctx \proves \wtt{\propB}{\Prop}}
		{\vctx \proves \wtt{\prop * \propB}{\Prop}}
\and
	\infer{\vctx \proves \wtt{\prop}{\Prop} \and \vctx \proves \wtt{\propB}{\Prop}}
		{\vctx \proves \wtt{\prop \wand \propB}{\Prop}}
\and
	\infer{
		\vctx, \var:\sort\to\Prop \proves \wtt{\pred}{\sort\to\Prop} \and
		\text{$\var$ is guarded in $\pred$}
	}{
		\vctx \proves \wtt{\MU \var. \pred}{\sort\to\Prop}
	}
\and
	\infer{\vctx, x:\sort \proves \wtt{\prop}{\Prop}}
		{\vctx \proves \wtt{\Exists x:\sort. \prop}{\Prop}}
\and
	\infer{\vctx, x:\sort \proves \wtt{\prop}{\Prop}}
		{\vctx \proves \wtt{\All x:\sort. \prop}{\Prop}}
\and
	\infer{
		\vctx \proves \wtt{\prop}{\Prop} \and
		\vctx \proves \wtt{\iname}{\textsort{InvName}}
	}{
		\vctx \proves \wtt{\knowInv{\iname}{\prop}}{\Prop}
	}
\and
	\infer{\vctx \proves \wtt{\melt}{\textsort{Monoid}}}
		{\vctx \proves \wtt{\ownGGhost{\melt}}{\Prop}}
\and
	\infer{\vctx \proves \wtt{\state}{\textsort{State}}}
		{\vctx \proves \wtt{\ownPhys{\state}}{\Prop}}
\and
	\infer{\vctx \proves \wtt{\prop}{\Prop}}
		{\vctx \proves \wtt{\always\prop}{\Prop}}
\and
	\infer{\vctx \proves \wtt{\prop}{\Prop}}
		{\vctx \proves \wtt{\later\prop}{\Prop}}
\and
	\infer{
		\vctx \proves \wtt{\prop}{\Prop} \and
		\vctx \proves \wtt{\mask}{\textsort{InvMask}} \and
		\vctx \proves \wtt{\mask'}{\textsort{InvMask}}
	}{
		\vctx \proves \wtt{\pvsA{\prop}{\mask}{\mask'}}{\Prop}
	}
\and
	\infer{
		\vctx \proves \wtt{\expr}{\textsort{Exp}} \and
		\vctx \proves \wtt{\pred}{\textsort{Val} \to \Prop} \and
		\vctx \proves \wtt{\mask}{\textsort{InvMask}}
	}{
		\vctx \proves \wtt{\dynA{\expr}{\pred}{\mask}}{\Prop}
	}
\and
	\infer{
		\vctx \proves \wtt{\prop}{\Prop}
	}{
		\vctx \proves \wtt{\timeless{\prop}}{\Prop}
	}
\end{mathparpagebreakable}


\section{Base logic}

The judgment $\vctx \mid \pfctx \proves \prop$ says that with free variables $\vctx$, proposition $\prop$ holds whenever all assumptions $\pfctx$ hold.
We implicitly assume that an arbitrary variable context, $\vctx$, is added to every constituent of the rules.
Axioms $\prop \Ra \propB$ stand for judgments $\vctx \mid \cdot \proves \prop \Ra \propB$ with no assumptions.
(Bi-implications are analogous.)

% \subsubsection{Judgments}
% 
% Proof rules implicitly assume well-sortedness.  

% e\subsection{Laws of intuitionistic higher-order logic with guarded recursion over a simply-typed lambda calculus}\label{sec:HOL}

This is entirely standard.

Soundness follows from the theorem that ${\cal U}(\any, \textdom{Prop})
: {\cal U}^{\textrm{op}} \to \textrm{Poset}$ is a hyperdoctrine. 

\begin{mathpar}
\inferH{Asm}
  {\prop \in \pfctx}
  {\pfctx \proves \prop}
\and
\inferH{Eq}
  {\pfctx \proves \prop(\term) \\ \pfctx \proves \term = \term'}
  {\pfctx \proves \prop(\term')}
\and
\infer[$\wedge$I]
  {\pfctx \proves \prop \\ \pfctx \proves \propB}
  {\pfctx \proves \prop \wedge \propB}
\and
\infer[$\wedge$EL]
  {\pfctx \proves \prop \wedge \propB}
  {\pfctx \proves \prop}
\and
\infer[$\wedge$ER]
  {\pfctx \proves \prop \wedge \propB}
  {\pfctx \proves \propB}
\and
\infer[$\vee$E]
  {\pfctx \proves \prop \vee \propB \\
   \pfctx, \prop \proves \propC \\
   \pfctx, \propB \proves \propC}
  {\pfctx \proves \propC}
\and
\infer[$\vee$IL]
  {\pfctx \proves \prop }
  {\pfctx \proves \prop \vee \propB}
\and
\infer[$\vee$IR]
  {\pfctx \proves \propB}
  {\pfctx \proves \prop \vee \propB}
\and
\infer[$\Ra$I]
  {\pfctx, \prop \proves \propB}
  {\pfctx \proves \prop \Ra \propB}
\and
\infer[$\Ra$E]
  {\pfctx \proves \prop \Ra \propB \\ \pfctx \proves \prop}
  {\pfctx \proves \propB}
\and
\infer[$\forall_1$I]
  {\pfctx, x : \sort \proves \prop}
  {\pfctx \proves \forall x: \sort.\; \prop}
\and
\infer[$\forall_1$E]
  {\pfctx \proves \forall X \in \sort.\; \prop \\
   \pfctx \proves \term: \sort}
  {\pfctx \proves \prop[\term/X]}
\and
\infer[$\exists_1$E]
  {\pfctx \proves \exists X\in \sort.\; \prop \\
   \pfctx, X : \sort, \prop \proves \propB}
  {\pfctx \proves \propB}
\and
\infer[$\exists_1$I]
  {\pfctx \proves \prop[\term/X] \\
   \pfctx \proves \term: \sort}
  {\pfctx \proves \exists X: \sort. \prop}
\and
\infer[$\forall_2$I]
  {\pfctx, \var: \Pred(\sort) \proves \prop}
  {\pfctx \proves \forall \var\in \Pred(\sort).\; \prop}
\and
\infer[$\forall_2$E]
  {\pfctx \proves \forall \var. \prop \\
   \pfctx \proves \propB: \Prop}
  {\pfctx \proves \prop[\propB/\var]}
\and
\infer[$\exists_2$E]
  {\pfctx \proves \exists \var \in \Pred(\sort).\prop \\
   \pfctx, \var : \Pred(\sort), \prop \proves \propB}
  {\pfctx \proves \propB}
\and
\infer[$\exists_2$I]
  {\pfctx \proves \prop[\propB/\var] \\
   \pfctx \proves \propB: \Prop}
  {\pfctx \proves \exists \var. \prop}
\and
\inferB[Elem]
  {\pfctx \proves \term \in (X \in \sort). \prop}
  {\pfctx \proves \prop[\term/X]}
\and
\inferB[Elem-$\mu$]
  {\pfctx \proves \term \in (\mu\var \in \Pred(\sort). \pred)}
  {\pfctx \proves \term \in \pred[\mu\var \in \Pred(\sort). \pred/\var]}
\end{mathpar}

\subsection{Axioms from the logic of (affine) bunched implications}
\begin{mathpar}
\begin{array}{rMcMl}
  \prop * \propB &\Lra& \propB * \prop \\
  (\prop * \propB) * \propC &\Lra& \prop * (\propB * \propC) \\
  \prop * \propB &\Ra& \prop
\end{array}
\and
\begin{array}{rMcMl}
  (\prop \vee \propB) * \propC &\Lra& 
    (\prop * \propC) \vee (\propB * \propC)  \\
  (\prop \wedge \propB) * \propC &\Ra& 
    (\prop * \propC) \wedge (\propB * \propC)  \\
  (\Exists x. \prop) * \propB &\Lra& \Exists x. (\prop * \propB) \\
  (\All x. \prop) * \propB &\Ra& \All x. (\prop * \propB) 
\end{array}
\and
\infer
  {\pfctx, \prop_1 \proves \propB_1 \and
   \pfctx, \prop_2 \proves \propB_2}
  {\pfctx, \prop_1 * \prop_2 \proves \propB_1 * \propB_2}
\and
\infer
  {\pfctx, \prop * \propB \proves \propC}
  {\pfctx, \prop \proves \propB \wand \propC}
\and
\infer
  {\pfctx, \prop \proves \propB \wand \propC}
  {\pfctx, \prop * \propB \proves \propC}
\end{mathpar}

\subsection{Laws for ghosts and physical resources}

\begin{mathpar}
\begin{array}{rMcMl}
\ownGGhost{\melt} * \ownGGhost{\meltB} &\Lra&  \ownGGhost{\melt \mtimes \meltB} \\
\TRUE &\Ra&  \ownGGhost{\munit}\\
\ownGGhost{\mzero} &\Ra& \FALSE\\
\multicolumn{3}{c}{\timeless{\ownGGhost{\melt}}}
\end{array}
\and
\begin{array}{c}
\ownPhys{\state} * \ownPhys{\state'} \Ra \FALSE \\
\timeless{\ownPhys{\state}}
\end{array}
\end{mathpar}

\subsection{Laws for the later modality}\label{sec:later}

\begin{mathpar}
\inferH{Mono}
  {\pfctx \proves \prop}
  {\pfctx \proves \later{\prop}}
\and
\inferhref{L{\"o}b}{Loeb}
  {\pfctx, \later{\prop} \proves \prop}
  {\pfctx \proves \prop}
\and
\begin{array}[b]{rMcMl}
  \later{\always{\prop}} &\Lra& \always{\later{\prop}} \\
  \later{(\prop \wedge \propB)} &\Lra& \later{\prop} \wedge \later{\propB}  \\
  \later{(\prop \vee \propB)} &\Lra& \later{\prop} \vee \later{\propB} \\
\end{array}
\and
\begin{array}[b]{rMcMl}
  \later{\All x.\prop} &\Lra& \All x. \later\prop \\
  \later{\Exists x.\prop} &\Lra& \Exists x. \later\prop \\
  \later{(\prop * \propB)} &\Lra& \later\prop * \later\propB
\end{array}
\end{mathpar}

\subsection{Laws for the always modality}\label{sec:always}

\begin{mathpar}
\axiomH{Necessity}
  {\always{\prop} \Ra \prop}
\and
\inferhref{$\always$I}{AlwaysIntro}
  {\always{\pfctx} \proves \prop}
  {\always{\pfctx} \proves \always{\prop}}
\and
\begin{array}[b]{rMcMl}
  \always(\term =_\sort \termB) &\Lra& \term=_\sort \termB \\
  \always{\prop} * \propB &\Lra& \always{\prop} \land \propB \\
  \always{(\prop \Ra \propB)} &\Ra& \always{\prop} \Ra \always{\propB} \\
\end{array}
\and
\begin{array}[b]{rMcMl}
  \always{(\prop \land \propB)} &\Lra& \always{\prop} \land \always{\propB} \\
  \always{(\prop \lor \propB)} &\Lra& \always{\prop} \lor \always{\propB} \\
  \always{\All x. \prop} &\Lra& \All x. \always{\prop} \\
  \always{\Exists x. \prop} &\Lra& \Exists x. \always{\prop} \\
\end{array}
\end{mathpar}
Note that $\always$ binds more tightly than $*$, $\land$, $\lor$, and $\Ra$.

\section{Program logic}\label{sec:proglog}

Hoare triples and view shifts are syntactic sugar for weakest (liberal) preconditions and primitive view shifts, respectively:
\[
\hoare{\prop}{\expr}{\Ret\val.\propB}[\mask] \eqdef \always{(\prop \Ra \dynA{\expr}{\lambda\Ret\val.\propB}{\mask})}
\qquad\qquad
\begin{aligned}
\prop \vs[\mask_1][\mask_2] \propB &\eqdef \always{(\prop \Ra \pvsA{\propB}{\mask_1}{\mask_2})} \\
\prop \vsE[\mask_1][\mask_2] \propB &\eqdef \prop \vs[\mask_1][\mask_2] \propB \land \propB \vs[\mask2][\mask_1] \prop
\end{aligned}
\]
We write just one mask for a view shift when $\mask_1 = \mask_2$.
The convention for omitted masks is generous:
An omitted $\mask$ is $\top$ for Hoare triples and $\emptyset$ for view shifts.

% PDS: We're repeating ourselves. We gave Γ conventions and we're about to give Θ conventions. Also, the scope of "Below" is unclear.
% Below, we implicitly assume the same context for all judgements which don't have an explicit context at \emph{all} pre-conditions \emph{and} the conclusion.

Henceforward, we implicitly assume a proof context, $\pfctx$, is added to every constituent of the rules.
Generally, this is an arbitrary proof context.
We write $\provesalways$ to denote judgments that can only be extended with a boxed proof context.

\ralf{Give the actual base rules from the Coq development instead}

\subsection{Hoare triples}
\begin{mathpar}
\inferH{Ret}
  {}
  {\hoare{\TRUE}{\valB}{\Ret\val. \val = \valB}[\mask]}
\and
\inferH{Bind}
  {\hoare{\prop}{\expr}{\Ret\val. \propB}[\mask] \\
   \All \val. \hoare{\propB}{K[\val]}{\Ret\valB.\propC}[\mask]}
  {\hoare{\prop}{K[\expr]}{\Ret\valB.\propC}[\mask]}
\and
\inferH{Csq}
  {\prop \vs \prop' \\
    \hoare{\prop'}{\expr}{\Ret\val.\propB'}[\mask] \\   
   \All \val. \propB' \vs \propB}
  {\hoare{\prop}{\expr}{\Ret\val.\propB}[\mask]}
\and
\inferH{Frame}
  {\hoare{\prop}{\expr}{\Ret\val. \propB}[\mask]}
  {\hoare{\prop * \propC}{\expr}{\Ret\val. \propB * \propC}[\mask \uplus \mask']}
\and
\inferH{AFrame}
  {\hoare{\prop}{\expr}{\Ret\val. \propB}[\mask] \and \text{$\expr$ not a value}
  }
  {\hoare{\prop * \later\propC}{\expr}{\Ret\val. \propB * \propC}[\mask \uplus \mask']}
% \and
% \inferH{Fork}
%   {\hoare{\prop}{\expr}{\Ret\any. \TRUE}[\top]}
%   {\hoare{\later\prop * \later\propB}{\fork{\expr}}{\Ret\val. \val = \textsf{fRet} \land \propB}[\mask]}
\and
\inferH{ACsq}
  {\prop \vs[\mask \uplus \mask'][\mask] \prop' \\
    \hoare{\prop'}{\expr}{\Ret\val.\propB'}[\mask] \\   
   \All\val. \propB' \vs[\mask][\mask \uplus \mask'] \propB \\
   \physatomic{\expr}
  }
  {\hoare{\prop}{\expr}{\Ret\val.\propB}[\mask \uplus \mask']}
\end{mathpar}

\subsection{View shifts}

\begin{mathpar}
\inferH{NewInv}
  {\infinite(\mask)}
  {\later{\prop} \vs[\mask] \exists \iname\in\mask.\; \knowInv{\iname}{\prop}}
\and
\inferH{FpUpd}
  {\melt \mupd \meltsB}
  {\ownGGhost{\melt} \vs \exists \meltB \in \meltsB.\; \ownGGhost{\meltB}}
\and
\inferH{VSTrans}
  {\prop \vs[\mask_1][\mask_2] \propB \and \propB \vs[\mask_2][\mask_3] \propC \and \mask_2 \subseteq \mask_1 \cup \mask_3}
  {\prop \vs[\mask_1][\mask_3] \propC}
\and
\inferH{VSImp}
  {\always{(\prop \Ra \propB)}}
  {\prop \vs[\emptyset] \propB}
\and
\inferH{VSFrame}
  {\prop \vs[\mask_1][\mask_2] \propB}
  {\prop * \propC \vs[\mask_1 \uplus \mask'][\mask_2 \uplus \mask'] \propB * \propC}
\and
\inferH{VSTimeless}
  {\timeless{\prop}}
  {\later \prop \vs \prop}
\and
\axiomH{InvOpen}
  {\knowInv{\iname}{\prop} \proves \TRUE \vs[\{ \iname \} ][\emptyset] \later \prop}
\and
\axiomH{InvClose}
  {\knowInv{\iname}{\prop} \proves \later \prop \vs[\emptyset][\{ \iname \} ] \TRUE }
\end{mathpar}

\vspace{5pt}
Note that $\timeless{\prop}$ means that $\prop$ does not depend on the step index.
Furthermore, $$\melt \mupd \meltsB \eqdef \always{\All \melt_f. \melt \sep \melt_f \Ra \Exists \meltB \in \meltsB. \meltB \sep \melt_f}$$

\subsection{Derived rules}

\paragraph{Derived structural rules.}
The following are easily derived by unfolding the sugar for Hoare triples and view shifts.
\begin{mathpar}
\inferHB{Disj}
  {\hoare{\prop}{\expr}{\Ret\val.\propC}[\mask] \and \hoare{\propB}{\expr}{\Ret\val.\propC}[\mask]}
  {\hoare{\prop \lor \propB}{\expr}{\Ret\val.\propC}[\mask]}
\and
\inferHB{VSDisj}
  {\prop \vs[\mask_1][\mask_2] \propC \and \propB \vs[\mask_1][\mask_2] \propC}
  {\prop \lor \propB \vs[\mask_1][\mask_2] \propC}
\and
\inferHB{Exist}
  {\All \var. \hoare{\prop}{\expr}{\Ret\val.\propB}[\mask]}
  {\hoare{\Exists \var. \prop}{\expr}{\Ret\val.\propB}[\mask]}
\and
\inferHB{VSExist}
  {\All \var. (\prop \vs[\mask_1][\mask_2] \propB)}
  {(\Exists \var. \prop) \vs[\mask_1][\mask_2] \propB}
\and
\inferHB{BoxOut}
  {\always\propB \provesalways \hoare{\prop}{\expr}{\Ret\val.\propC}[\mask]}
  {\hoare{\prop \land \always{\propB}}{\expr}{\Ret\val.\propC}[\mask]}
\and
\inferHB{VSBoxOut}
  {\always\propB \provesalways \prop \vs[\mask_1][\mask_2] \propC}
  {\prop \land \always{\propB} \vs[\mask_1][\mask_2] \propC}
 \and
\inferH{False}
  {}
  {\hoare{\FALSE}{\expr}{\Ret \val. \prop}[\mask]}
\and
\inferH{VSFalse}
  {}
  {\FALSE \vs[\mask_1][\mask_2] \prop }
\end{mathpar}
The proofs all follow the same pattern, so we only show two of them in detail.
\begin{proof}[Proof of \ruleref{Exist}]
	After unfolding the syntactic sugar for Hoare triples and removing the boxes from premise and conclusion, our goal becomes
	\[
		(\Exists \var. \prop(\var)) \Ra \dynA{\expr}{\Lam\val. \propB}{\mask}
	\]
	(remember that $\var$ is free in $\prop$) and the premise reads
	\[
		\All \var. \prop(\var) \Ra \dynA{\expr}{\Lam\val. \propB}{\mask}.
	\]
	Let $\var$ be given and assume $\prop(\var)$.
	To show $\dynA{\expr}{\Lam\val. \propB}{\mask}$, apply the premise to $\var$ and $\prop(\var)$.
 
	For the other direction, assume
	\[
		\hoare{\Exists \var. \prop(\var)}{\expr}{\Ret\val. \propB}[\mask]
	\]
	and let $\var$ be given.
	We have to show $\hoare{\prop(\var)}{\expr}{\Ret\val. \propB}[\mask]$.
	This trivially follows from \ruleref{Csq} with $\prop(\var) \Ra \Exists \var. \prop(\var)$.
\end{proof}

\begin{proof}[Proof of \ruleref{BoxOut}]
  After unfolding the syntactic sugar for Hoare triples, our goal becomes
  \begin{equation}\label{eq:boxin:goal}
    \always\pfctx \proves \always\bigl(\prop\land\always \propB \Ra \dynA{\expr}{\Lam\val. \propC}{\mask}\bigr)
  \end{equation}
  while our premise reads
  \begin{equation}\label{eq:boxin:as}
    \always\pfctx, \always\propB \proves \always(\prop \Ra \dynA{\expr}{\Lam\val. \propC}{\mask})
  \end{equation}
  By the introduction rules for $\always$ and implication, it suffices to show
  \[  (\always\pfctx), \prop,\always \propB \proves \dynA{\expr}{\Lam\val. \propC}{\mask} \]
  By modus ponens and \ruleref{Necessity}, it suffices to show~\eqref{eq:boxin:as}, which is exactly our assumption.
  
  For the other direction, assume~\eqref{eq:boxin:goal}. We have to show~\eqref{eq:boxin:as}. By \ruleref{AlwaysIntro} and implication introduction, it suffices to show
  \[  (\always\pfctx), \prop,\always \propB \proves \dynA{\expr}{\Lam\val. \propC}{\mask} \]
  which easily follows from~\eqref{eq:boxin:goal}.
\end{proof}

\paragraph{Derived rules for invariants.}
Invariants can be opened around atomic expressions and view shifts.

\begin{mathpar}
\inferH{Inv}
  {\hoare{\later{\propC} * \prop }
          {\expr}
          {\Ret\val. \later{\propC} * \propB }[\mask]
          \and \physatomic{\expr}
  }
  {\knowInv{\iname}{\propC} \proves \hoare{\prop}
          {\expr}
          {\Ret\val. \propB}[\mask \uplus \{ \iname \}]
  }
\and
\inferH{VSInv}
  {\later{\prop} * \propB \vs[\mask_1][\mask_2] \later{\prop} * \propC}
  {\knowInv{\iname}{\prop} \proves \propB \vs[\mask_1 \uplus \{ \iname \}][\mask_2 \uplus \{ \iname \}] \propC}
\end{mathpar}

\begin{proof}[Proof of \ruleref{Inv}]
  Use \ruleref{ACsq} with $\mask_1 \eqdef \mask \cup \{\iname\}$, $\mask_2 \eqdef \mask$.
  The view shifts are obtained by \ruleref{InvOpen} and \ruleref{InvClose} with framing of $\mask$ and $\prop$ or $\propB$, respectively.
\end{proof}

\begin{proof}[Proof of \ruleref{VSInv}]
Analogous to the proof of \ruleref{Inv}, using \ruleref{VSTrans} instead of \ruleref{ACsq}.
\end{proof}

\subsubsection{Unsound rules}

Some rule suggestions (or rather, wishes) keep coming up, which are unsound. We collect them here.
\begin{mathpar}
	\infer
	{P \vs Q}
	{\later P \vs \later Q}
	\and
	\infer
	{\later(P \vs Q)}
	{\later P \vs \later Q}
\end{mathpar}

Of course, the second rule implies the first, so let's focus on that.
Since implications work under $\later$, from $\later P$ we can get $\later \pvs{Q}$.
If we now try to prove $\pvs{\later Q}$, we will be unable to establish world satisfaction in the new world:
We have no choice but to use $\later \pvs{Q}$ at one step index below what we are operating on (because we have it under a $\later$).
We can easily get world satisfaction for that lower step-index (by downwards-closedness of step-indexed predicates).
We can, however, not make much use of the world satisfaction that we get out, becaase it is one step-index too low.

\subsection{Adequacy}

The adequacy statement reads as follows:
\begin{align*}
 &\All \mask, \expr, \val, \pred, i, \state, \state', \tpool'.
 \\&( \proves \hoare{\ownPhys\state}{\expr}{x.\; \pred(x)}[\mask]) \implies
 \\&\cfg{\state}{[i \mapsto \expr]} \step^\ast
     \cfg{\state'}{[i \mapsto \val] \uplus \tpool'} \implies
     \\&\pred(\val)
\end{align*}
where $\pred$ can mention neither resources nor invariants.

\subsection{Axiom lifting}\label{sec:lifting}

The following lemmas help in proving axioms for a particular language.
The first applies to expressions with side-effects, and the second to side-effect-free expressions.
\dave{Update the others, and the example, wrt the new treatment of $\predB$.}
\begin{align*}
 &\All \expr, \state, \pred, \prop, \propB, \mask. \\
 &\textlog{reducible}(e) \implies \\
 &(\All \expr', \state'. \cfg{\state}{\expr} \step \cfg{\state'}{\expr'} \implies \pred(\expr', \state')) \implies \\
 &{} \proves \bigl( (\All \expr', \state'. \pred (\expr', \state') \Ra \hoare{\prop}{\expr'}{\Ret\val. \propB}[\mask]) \Ra \hoare{ \later \prop * \ownPhys{\state} }{\expr}{\Ret\val. \propB}[\mask] \bigr) \\
 \quad\\
 &\All \expr, \pred, \prop, \propB, \mask. \\
 &\textlog{reducible}(e) \implies \\
 &(\All \state, \expr_2, \state_2. \cfg{\state}{\expr} \step \cfg{\state_2}{\expr_2} \implies \state_2 = \state \land \pred(\expr_2)) \implies \\
 &{} \proves \bigl( (\All \expr'. \pred(\expr') \Ra \hoare{\prop}{\expr'}{\Ret\val. \propB}[\mask]) \Ra \hoare{\later\prop}{\expr}{\Ret\val. \propB}[\mask] \bigr)
\end{align*}
Note that $\pred$ is a meta-logic predicate---it does not depend on any world or resources being owned.

The following specializations cover all cases of a heap-manipulating lambda calculus like $F_{\mu!}$.
\begin{align*}
 &\All \expr, \expr', \prop, \propB, \mask. \\
 &\textlog{reducible}(e) \implies \\
 &(\All \state, \expr_2, \state_2. \cfg{\state}{\expr} \step \cfg{\state_2}{\expr_2} \implies \state_2 = \state \land \expr_2 = \expr') \implies \\
 &{} \proves (\hoare{\prop}{\expr'}{\Ret\val. \propB}[\mask] \Ra \hoare{\later\prop}{\expr}{\Ret\val. \propB}[\mask] ) \\
 \quad \\
 &\All \expr, \state, \pred, \mask. \\
 &\textlog{atomic}(e) \implies \\
 &\bigl(\All \expr_2, \state_2. \cfg{\state}{\expr} \step \cfg{\state_2}{\expr_2} \implies \pred(\expr_2, \state_2)\bigr) \implies \\
 &{} \proves (\hoare{ \ownPhys{\state} }{\expr}{\Ret\val. \Exists\state'. \ownPhys{\state'} \land \pred(\val, \state') }[\mask] )
\end{align*}
The first is restricted to deterministic pure reductions, like $\beta$-reduction.
The second is suited to proving triples for (possibly non-deterministic) atomic expressions; for example, with $\expr \eqdef \;!\ell$ (dereferencing $\ell$) and $\state \eqdef h \mtimes \ell \mapsto \valB$ and $\pred(\val, \state') \eqdef \state' = (h \mtimes \ell \mapsto \valB) \land \val = \valB$, one obtains the axiom $\All h, \ell, \valB. \hoare{\ownPhys{h \mtimes \ell \mapsto \valB}}{!\ell}{\Ret\val. \val = \valB \land \ownPhys{h \mtimes \ell \mapsto \valB} }$.
%Axioms for CAS-like operations can be obtained by first deriving rules for the two possible cases, and then using the disjunction rule.

%%% Local Variables:
%%% mode: latex
%%% TeX-master: "iris"
%%% End:
